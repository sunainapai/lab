\documentclass{article}
\usepackage[a4paper]{geometry}
\usepackage{parskip}
\usepackage[colorlinks=true,linkcolor=black,urlcolor=blue]{hyperref}
\usepackage{amsmath}
\usepackage{tikz}
\urlstyle{same}
\date{}
\author{Sunaina Pai}

\title{Folded Sheet of Paper}
\begin{document}
\maketitle

\section*{Problem}
A rectangular sheet of paper is folded so that two diagonally opposite
corners come together. If the crease formed is the same length as the
longer side of the sheet, what is the ratio of the longer side of the
sheet to the shorter side?

Source: \url{http://www.qbyte.org/puzzles/puzzle01.html#p1}

\section*{Solution}
Let us represent the sheet of paper as a rectangle \( ABCD \) where
\( AB > BC \). Let us represent the crease with the line segment
\( EF \) such that the crease \( EF \) meets \( AB \) and \( DC \) at
\( E \) and \( F \), respectively. Let \( H \) be a point on \( DC \)
such that \( EH \perp DC \).

\begin{figure}[htbp]
\centering
\begin{tikzpicture}
\coordinate (A) at (0, 4);
\coordinate (B) at (5, 4);
\coordinate (C) at (5, 0);
\coordinate (D) at (0, 0);
\coordinate (E) at (4, 4);
\coordinate (F) at (1, 0);
\coordinate (H) at (4, 0);
\node[above left] at (A) {A};
\node[above right] at (B) {B};
\node[below right] at (C) {C};
\node[below left] at (D) {D};
\node[above] at (E) {E};
\node[below] at (F) {F};
\node[below] at (H) {H};
\draw (A) -- (B) -- (C) -- (D) -- (A);
\draw (E) -- (F) node[midway,below right]{b};
\draw[densely dotted] (A) -- (F) node[midway,above right]{p};
\draw[densely dotted] (E) -- (H);
\draw[<->] ([xshift=+0.5cm]B) -- ([xshift=+0.5cm]C) node[midway,right]{a};
\draw[<->] ([yshift=-0.5cm]F) -- ([yshift=-0.5cm]C) node[midway,below]{p};
\draw[<->] ([yshift=-1.0cm]D) -- ([yshift=-1.0cm]C) node[midway,below]{b};
\end{tikzpicture}
\end{figure}


Let us define some of the line segments with single letter variables:
\begin{align*}
AB & = DC = b, \\
AD & = BC = a, \\
AF & = FC = p.
\end{align*}
Note that when the sheet is folded to meet at corners \( A \) and
\( C \), \( FC \) coincides with \( AF \). This justifies \( AF = FC \).

Since the length of the crease is same as longer side, we get
\[
EF = b.
\]

We compute \( DF \) as
\begin{equation}
\label{df}
DF = b - p.
\end{equation}

By applying Pythagorean theorem to the right-angled \( \triangle ADF \),
we get
\begin{align*}
AF^2 = AD^2 + DF^2
& \implies p^2  = a^2 + (b - p)^2 \\
& \iff     p^2  = a^2 + b^2 + p^2 - 2bp \\
& \iff     p    = \frac{a^2 + b^2}{2b}.
\end{align*}

If we turn the rectangle upside down, the problem remains the same, i.e.
\( EB \) appears in the modified problem where \( DF \) is in the
original problem. Therefore by symmetry and \eqref{df},
\[
DF = EB = HC = b - p.
\]

Now we compute \( FH \) as
\begin{align*}
FH & = DC - DF - HC \\
   & = b - 2(b - p) \\
   & = b - 2b + 2p \\
   & = 2p - b \\
   & = \frac{2 \left( a^2 + b^2 \right)}{2b} - b \\
   & = \frac{a^2 + b^2}{b} - b \\
   & = \frac{a^2 + b^2 - b^2}{b} \\
   & = \frac{a^2}{b}.
\end{align*}

By applying Pythagorean theorem to the right-angled \( \triangle EFH \)
we get
\begin{align}
EH^2 + FH^2 = EF^2 & \implies a^2 + \left( \frac{a^2}{b} \right)^2 = b^2
                     \nonumber \\
                   & \iff b^4 - a^2 b^2 - a^4 = 0.
                     \label{eq2}
\end{align}

Let \( \left( \frac{b}{a} \right)^2 = x \). Then \( b^2 = a^2 x \).
Substituting this in \eqref{eq2} we get
\begin{align*}
b^4 - a^2 b^2 - a^4 = 0 & \iff a^4 x^2 - a^4 x - a^4 = 0 \\
                        & \iff a^4 (x^2 - x - 1) = 0.
\end{align*}
Since \( a \neq 0 \), we have \( x^2 - x - 1 = 0 \). Therefore
\[
x = \frac{1 \pm \sqrt{1 + 4}}{2} = \frac{1 \pm \sqrt{5}}{2}.
\]
Since \( x > 0 \), we ignore the negative result for \( x \) and we get
\begin{align*}
x = \frac{1 + \sqrt{5}}{2}
& \iff \left( \frac{b}{a} \right)^2 = \frac{1 + \sqrt{5}}{2} \\
& \iff \frac{b}{a} = \sqrt{\frac{1 + \sqrt{5}}{2}}.
\end{align*}

\end{document}
